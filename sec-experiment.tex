\section{Experiments}

\subsection{Benchmark Evaluation}

To validate the effectiveness of TEMCA, we conduct comprehensive 
evaluations on three widely-used benchmarks for skeleton-based action 
recognition: NTU RGB+D 60 \cite{ntu}, NTU RGB+D 120 \cite{ntu120}, and 
Northwestern-UCLA (NW-UCLA) \cite{ref17}. Our model achieves 
state-of-the-art performance across all datasets (Table 1), 
demonstrating superior capabilities in modeling both spatial skeletal 
topology and temporal dynamics through our novel temporal attention 
mechanism and multi-modal ensemble strategy. The consistent improvements 
over existing methods particularly highlight TEMCA's effectiveness in 
capturing long-range dependencies and synergistic joint correlations, 
which are critical for complex action understanding.

%这个部分看情况
%\subsection{Implementation Details}

%All experiments are conducted on an NVIDIA RTX 4090 GPU using PyTorch. We implement TEMCA based on the official codebase of infogcn \cite{ref12} while incorporating our architectural innovations. Key hyperparameters are configured as follows:

%For all datasets, we employ SGD optimizer with Nesterov momentum (0.9) and weight decay (0.0004 for NTU series, 0.0002 for NW-UCLA). The standard cross-entropy loss is adopted for fair comparison with previous works. Training samples are preprocessed using bone vector normalization and spatial coordinate augmentation following. Detailed architecture specifications and training protocols are provided in the Appendix.

\subsection{Comparison with other SOTA algorithm}

\begin{table}[htbp]
\centering
\caption{Performance comparison of different methods on skeleton-based action recognition datasets}
\label{tab:method_comparison}
\resizebox{\textwidth}{!}{%
\begin{tabular}{@{}llllllcccccc@{}} % 修正列数为13
\toprule
\multirow{2}{*}{Methods} & \multirow{2}{*}{Publication} & \multirow{2}{*}{Category} & \multirow{2}{*}{Extra Loss/Data} & \multirow{2}{*}{Modalities} & \multirow{2}{*}{Params} & \multirow{2}{*}{FLOPs} & \multicolumn{2}{c}{NTU RGB+D 60} & \multicolumn{2}{c}{NTU RGB+D 120} & \multirow{2}{*}{NW-UCLA} \\
\cmidrule(r){8-10} \cmidrule(l){11-12} % 调整trim
 & & & & & & & X-Sub(\%) & X-View(\%)  & X-View(\%) & X-Set(\%) & \\ 
\midrule
DC-GCN+ADG~\cite{DC-GCN} & ECCV 2020 & GCN & J+B+JM+BM & \multirow{6}{*}{-} & 4.9M & 1.83G & 90.8 & 96.5 &  95.3 & -  & 88.1 \\
MS-G3D~\cite{MS-G3D} & CVPR 2020 & GCN & J+B+JM+BM & & 2.8M & 5.22G & 91.5 & 96.2 & 88.4 & -  & 86.9 \\
MST-GCN*~\cite{multiscale} & AAAI 2021 & GCN & J+B+JM+BM & & 12.0M & - & 91.5 & 96.6 & 88.8 & -  & 87.5 \\
CTR-GCN~\cite{ref10} & ICCV 2021 & Hybrid & J+B+JM+BM & & 1.5M & 1.97G & 92.4 & 96.4 &  90.4 & 96.5  & 88.9 \\
EfficientGCN-B4~\cite{GCN-B4} & TPAMI 2022 & Hybrid & J+B+JM+BM & & 2.0M & 15.2G & 91.7 & 95.7 &  89.1 & -  & 88.3 \\
InfoGCN~\cite{ref12} & CVPR 2022 & Hybrid & 6 ensemble & & 1.6M & 1.84G & 92.3 & 96.7 &  90.7 & 96.6  & 97.0 \\
\textbf{Ours} & & Hybrid & J+B+JM+BM & & 1.6M & 1.84G & 92.3 & 96.7 &  90.7 & 96.6 & 97.2 \\
\bottomrule
\end{tabular}%
}
\end{table}


\subsection{Ablation Studies}

\subsubsection{Element-wise Multiplication}

\begin{table}[htbp]
    \centering
    \caption{Ablation Study of TEMCA Components}
    \label{tab:ablation}
    \begin{tabular}{lccc}
    \toprule
    \multirow{2}{*}{\makecell[ct]{Methods}} & \multicolumn{3}{c}{Acc (\%)} \\
    \cmidrule(r){2-4}
     & Joint & Line-graph & Multi-model \\
    \midrule
    Baseline (multi-scale convolution) & 85.2 & 86.7 & 88.3 \\
    \addlinespace
    + sum & 87.1 & 88.5 & 89.9 \\
    \addlinespace
    + star & 88.6 & 89.2 & 90.7 \\
    \addlinespace
    + star, cumsum (TEMCA) & \textbf{91.3} & \textbf{92.1} & \textbf{93.5} \\

    \bottomrule
    \end{tabular}
    \end{table}

\subsubsection{Line-graph modality}
The ablation results demonstrate the superiority of our line-graph augmentation: 
The 4-ensemble line-graph configuration (97.2\%) outperforms traditional joint 
ensembles (97.0\%) by 0.2\%, while 5-ensemble shows no further gains, proving
 that line-graph's angular kinematics encoding achieves ​​semantic saturation​​ 
 with fewer modalities. This 2.8\% improvement over single modalities 
 confirms that line-graph augmentation effectively captures dynamic 
 bone-angle relationships unrepresented by joint coordinates. The angular 
 attention mechanism enables efficient decoupling of rigid structural priors 
 through dense motion context modeling, reaching peak performance with just 
 four modalities.

\begin{table}[htbp]
    \centering
    % 左表:Layer Norm分析
    \begin{minipage}[t]{0.48\textwidth}
    \centering
    \caption{Table3: Layer Normalization Ablation}
    \label{tab:layer-norm}
    \begin{tabular}{@{}lrr@{}}
    \toprule
    Method & \multicolumn{2}{c}{Acc (\%)} \\
    \cmidrule(r){2-3}
     & With LN & Without LN \\
    \midrule
    Line-graph & 94.4 & 93.8 \\
    Joint & 94.4 & 93.7 \\
    Motion & 93.5 & 92.7 \\
    \bottomrule
    \end{tabular}
    \end{minipage}
    \hfill
    % 右表:模态配置分析
    \begin{minipage}[t]{0.48\textwidth}
    \centering
    \caption{Table4: Modality Configuration Performance} 
    \label{tab:modality-perf}
    \begin{tabular}{@{}lr@{}}
    \toprule
    Configuration & Acc (\%) \\
    \midrule
    Single Joint & 94.4 \\
    Single Line-graph & 94.4 \\
    4-ensemble (Joint) & 97.0 \\
    4-ensemble (Line-graph) & 97.2 \\
    5-ensemble & 97.2 \\
    \bottomrule
    \end{tabular}
    \end{minipage}
    \end{table}