\section{Related Works}

\subsection{GCN for Skeleton-based Recognition}

Early approaches for skeleton-based action recognition primarily 
relied on Recurrent Neural Networks (RNNs) \cite{ref4,ref5} to model 
temporal dependencies and Convolutional Neural Networks (CNNs) 
\cite{ref6} to process joint coordinates. While these methods achieved 
reasonable performance, they struggled to capture the intrinsic spatial 
relationships between joints.

Since GCNs \cite{ref7,ref8} have emerged as the dominant 
paradigm due to their ability to aggregate topology. Recent 
advancements can be categorized into three directions:

\subsubsection{Learnable Adjacency Matrix}: Early GCN implementations
 \cite{ref7} relied on fixed topologies defined by natural bone 
 connectivity, which limited their capacity to model synergistic joint 
 correlations. Subsequent works \cite{ref8,ref9,ref10,ref11,re12,ref14,ref15,ref16} introduced learnable 
 adjacency matrices to capture adaptive long-range dependencies beyond 
 anatomical priors.

\subsubsection{Temporal Modeling}: While spatial graph convolutions 
have been extensively studied, temporal feature extraction remains 
underexplored. Most approaches employ the multi-scale 
temporal convolution modules \cite{ref9,ref10,ref11,ref12,ref13} or Use cross-frame 
interaction convolutions \cite{ref15}, where stacked local temporal convolutions 
struggle to model prolonged action dynamics. Our work addresses this by
firstly introducing attention mechanisms in the temporal graph domain. The 
proposed interval-sampled multi-scale attention dynamically weighs 
feature importance across time steps, proved to capturing complex 
interdependencies in multivariate skeleton sequences effectively. 

%\cite{temporal-attention-useful}.这一篇引用放到后面去

\subsubsection{Multi-modal Ensemble}: Shi et al. \cite{ref13} first 
demonstrated that training separate models on distinct skelenton
modalities, like joint or bone vectors, enhances robustness 
through complementary semantics. Follow-up works \cite{ref9,ref11,ref12} 
further validated the benefits of integrating modalities like motion 
vectors. Building on this paradigm, we propose line-graph 
augmentation—a novel modality that explicitly encodes angular 
kinematics between adjacent bones, enabling richer ensemble representations.

\subsection{Linear Self-attention}